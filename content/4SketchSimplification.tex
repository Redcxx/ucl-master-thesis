\chapter{Sketch Simplification}

Sketch Simplification refers to automatically converting rough sketches into simplified clean drawings. It is one of the most fundamental tasks because it is the first step in the whole pipeline. An accurate and fast method is required to provide a solid foundation for subsequent processing. However, as we will see, it is a relatively under-explored area and we have yet to see a fast and high-quality procedure for this task. Nevertheless, the task itself is relatively straight forward and similar problems have developed mature solutions, therefore it is not expected to be difficult. I will explain the recent research and main challenge and possible future direction in the following sections.

\label{chapterlabel4}
\section{Approaches \& Methods}
Traditionally, rough sketches are iteratively refined by the artist himself/herself. This requires manually tracing the rough sketch repeatedly to produce a clean and satisfying drawing. This process is tedious, time-consuming and involves a large overhead. There have been some methods proposed to simplify sketch drawings. Some assists user to clean up sketches based on geometry relationships between strokes\cite{fiserShipShapeDrawingBeautification2015} and fitting stroke using Bezier Curves\cite{baeILoveSketchAsnaturalaspossibleSketching2008}; Others simplify rough lines by removing unnecessary ones\cite{liuClosureawareSketchSimplification2015}. However, these traditional methods neither change the way artists clean up sketches nor provide a fully automatic way to simplify them. Moreover, they often operate on vector graphics, which is more dominant in the design industry and not the 2D animation industry.

As far as I know, the first attempt to simplify sketches in a fully automatic way on raster images was in 2016\cite{simo-serraLearningSimplifyFully2016}. The researchers used an encoder-decoder CNN architecture and achieved reasonable results. Worth mentioning, they used a loss map which assigns a higher weighting to lines and achieved much faster training. Their results showed that for simple sketches the model can produce accurate strokes (see figure \ref{fig:learning2simp}). Follow-up research approached this problem with a simple cGAN model and achieved slightly better results\cite{simo-serraMasteringSketchingAdversarial2017}, and was able to handle the inverse problem by converting clean sketch to rough sketch. Both attempts only trained with 68 pairs of data, which is insufficient, prone to bias, and does not prove the generalization capability of the model. But we have no way of knowing because they did not disclose the dataset used, and there are no other known attempts  to convert rasterized rough sketch to rasterized clean sketch directly that have a similar context to this master project.

\begin{figure}
    \centering
    \includegraphics[width=0.75\textwidth]{images/sketch/learning2simp.png}
    \caption{Visualization of the output image as training proceeds for encoder-decoder CNN model.} 
    \label{fig:learning2simp}
\end{figure}

All aspects of the model: preprocessing, augmentation, architecture and training etc. are simple without many modern techniques, however, it is able to generate relatively precise results. This showed that this task is comparatively simpler than others. However, the main challenge lay in this task is the training data. It is difficult to find high-quality training samples and papers either do not disclose the dataset they use or it is not suitable for our task. For example, there exists a benchmark dataset for sketch simplification\cite{yanBenchmarkRoughSketch2020}. However, this dataset focus on removing shadows and composition guidelines from the sketch image, rather than converting rough lines into clean lines.

\section{Design \& Implementation}
