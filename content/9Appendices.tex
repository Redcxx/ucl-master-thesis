\phantomsection
\addcontentsline{toc}{chapter}{Appendices}

% The \appendix command resets the chapter counter, and changes the chapter numbering scheme to capital letters.
%\chapter{Appendices}
\appendix
\chapter{Machine Learning}
\label{app:ml}
\section{Symmetry Problem}
\label{app:ml:sym}
In the context of weight initialization, symmetry problem is an argument for weights should not be initialized as same value. This is because if the weights are equal, then its gradient will be equal, and the weights are going to be updated by the same amount. If these weights are attached to the same neuron, it will theoretically remain the same throughout training, and therefore its capability will decrease.

\section{Knowledge Distillation}
\label{app:ml:kd}
While large models such as very deep neural network can have large capacity, it may not be fully utilized. This means that we can potentially reduce the size of the model while achieve similar results. This process is called knowledge distillation.

% \chapter{Another Appendix About Things}
% \label{appendixlabel2}
% (things)

% \chapter{Colophon}
% \label{appendixlabel3}
% \textit{This is a description of the tools you used to make your thesis. It helps people make future documents, reminds you, and looks good.}

% \textit{(example)} This document was set in the Times Roman typeface using \LaTeX\ and Bib\TeX , composed with a text editor. 
 % description of document, e.g. type faces, TeX used, TeXmaker, packages and things used for figures. Like a computational details section.
% e.g. http://tex.stackexchange.com/questions/63468/what-is-best-way-to-mention-that-a-document-has-been-typeset-with-tex#63503

% Side note:
%http://tex.stackexchange.com/questions/1319/showcase-of-beautiful-typography-done-in-tex-friends
