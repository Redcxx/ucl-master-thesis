% I may change the way this is done in a future version, 
%  but given that some people needed it, if you need a different degree title 
%  (e.g. Master of Science, Master in Science, Master of Arts, etc)
%  uncomment the following 3 lines and set as appropriate (this *has* to be before \maketitle)
% \makeatletter
% \renewcommand {\@degree@string} {Master of Things}
% \makeatother

\title{Machine Learning for 2D Animation}
\author{Weilue Luo}
\department{Department of Computer Science}

\maketitle
\makedeclaration

\begin{abstract} % 300 word limit
This preceding project researches and implements state-of-the-art (SOTA) methods in machine learning (ML) image-to-image translation for enhancing 2D animation development workflow. It tries to address the waterfall development problem by allowing artists to preview final outcome given initial creative inputs. Combining the field of frame interpolation, image colorization, sketch simplification, and super-resolution, this work indicates possible future directions and limitations for future iterations and further studies.
\end{abstract}

\begin{impactstatement}

	In 2D animation industry, the process of creating 2D animation consists of a number of steps, some of which could be labor intensive and often constitute up to 50\% of the budget. The industry is falling behind when it comes to creative tools in 2D animation. Currently, the lead artist gives front-loaded creative inputs and followed by intensive labor tasks that he has little control. We want the artist to have an accurate preview of what might be in the final product. Contrasting to traditional approaches, out proposed workflow is designed around the capabilities of the artists, allowing them to take a shot from start to finish with creative control along the way. The final tool-set will allow one animator to complete work that typically requires five animators to do.


\end{impactstatement}

\begin{acknowledgements}
This work is supervised by Dr. Tobias Ritschel, Professor of Computer Graphics, University College London and collaborated with NoGhost, a real-time creative animation studio based in London.
\end{acknowledgements}

\setcounter{tocdepth}{2} 
% Setting this higher means you get contents entries for
%  more minor section headers.

\tableofcontents
\listoffigures
\listoftables

